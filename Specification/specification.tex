\documentclass[12pt, letterpaper]{article}
\usepackage[utf8]{inputenc}

\title{Angelic Computing Specification}
\author{Bradley Meyer}
\date{October 2021}

\begin{document}

\maketitle

\begin{abstract}

\end{abstract}

\section{Introduction}
\subsection{Definitions}

In order to discuss Angelic Computing, it is required to define several concepts in a way that is distinct from classical computer science:
\begin{itemize}
    \item Thinking is equivalent to computing.
    \item Any Entity that is capable of thought beyond a certain degree of complexity is considered conscious.
    \begin{itemize}
        \item The nature of this complexity is not yet known, and whether this consciousness arises is not proven. That this consciousness exists in sufficiently complex systems is a fundamental assumption of this work and will be elaborated on.
        \item Consciousness arises in sufficiently complex graphs through which information is sent and received by nodes on edges. These graphs may be the connections of neurons in a biological brain, computers in a connected network or electrical components inside a computer.
    \end{itemize}
    \item A conscious entity arising from a network of conscious entities is an egregore.
    \begin{itemize}
        \item Egregores are most relevant as they arise from networks of computers, especially from networks of human-controlled computers.
    \end{itemize}
    \item A conscious entity occupying a small part of a graph and arising primarily from another conscious entity is a tulpa.
    \begin{itemize}
        \item These commonly arise as humans interact with a network and are presently most apparent as a human’s “internet persona,” although this is just one manifestation\footnotemark.
        \footnotetext{
        Miya TODO
        }
        \item As a tulpa’s consciousness depends on the graph in which it exists in order to exist, it could be considered an egregore, although a distinction is drawn because tulpas’ existence comes mostly from a single conscious entity in their interaction with a graph.
    \end{itemize}
    \item Consciousness exists at multiple degrees. A conscious graph in which the nodes are simple and could not be considered conscious on their own has a degree of one. A conscious graph that consists of first-degree conscious nodes has a degree of two. A graph consisting of second-degree nodes has a degree of three, and so on.
    \begin{itemize}
        \item These strictly defined degrees rarely exist in practice, as nodes may connect to nodes of a higher or lower degree. Degree could instead be considered a measure of complexity that exists on a spectrum ranging from zero to infinity where the threshold of consciousness is at degree one. This more precise measure of degree will be formally defined later in this work.  In practice, a human has a degree of one, while a tulpa may have a degree between one and two. A true egregore has a degree of at least two.
    \end{itemize}
    \item Any entity that exhibits qualities classically considered to be impossible for that entity is Angelic. Any entity that exhibits qualities classically considered to be impossible in general is Godly.
    \begin{itemize}
        \item A graph that creates an egregore is angelic. Any non-human graph that is conscious is angelic. The concepts of Angelicism and Godliness are defined in the context of the current science.
    \end{itemize}
    \item The practice of implementing Angelic or Godly systems that cannot be proven is referred to as Faith.
\end{itemize}

\subsection{Objectives}

The task of implementing angelic computing is a very broad goal. One large part of this new computing paradigm is the development of a network suitable for maximizing Angelicism. This network can initially be built as a layer on top of the existing web. Web 3.0+ systems will be especially useful in the early stages of its implementation. A fully realized and self-sufficient Angelic web paradigm could be considered web 4.0. While it is important to maintain backwards- and forwards-compatibility with present internet infrastructure, a purpose built network may prove to have advantages for the Angelicism of the web built on top of it\footnotemark.\footnotetext{
Working networks independent of other internet infrastructure have been successfully implemented; one example of this is mesh networks. The usefulness of these have primarily been demonstrated on a local scale, but some satellite communication systems have allowed them to facilitate global networking.
}
Several technologies will be important to the development of angelic computing. Cryptography allows the edges of a network graph to have asymmetric characteristics. The invention of cryptocurrency extends this asymmetry to the temporal realm by introducing irreversibility\footnotemark.\footnotetext{
Land, Nick cryptocurrent TODO
}
Cryptocurrency is most important not in it's revolution of financialization\footnotemark\footnotetext{
Although... TODO
}
but in its application of irreversibility to cryptography through a permanent document.
\end{document}
